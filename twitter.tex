\documentclass{article}
\usepackage{graphicx}
\usepackage{natbib}  

\title{Verhaltensanalyse leichtgläubiger Twitter-Nutzer}
\author{Ismail Soussi}
\date{\today}

\begin{document}
	
	\begin{figure}
		\centering
		\includegraphics[width=0.6\textwidth]{C:/Users/ismai/Desktop/twitter.png}
		\caption{twitter}
		\label{fig:twitter}
	\end{figure}
	
	\maketitle
	
	\begin{abstract}
		Hier geben Sie eine kurze Zusammenfassung Ihrer Arbeit an.
	\end{abstract}
	
	\section{Einleitung}
	In der Einleitung führen Sie in Ihr Thema ein, erläutern den Kontext und Ihre Forschungsfrage. Die vorliegende Mini-Arbeit untersucht die Verhaltensanalyse leichtgläubiger Twitter-Nutzer. Das Ziel ist es, die Einflüsse und Faktoren zu identifizieren, die das Verhalten dieser Nutzer auf der Plattform beeinflussen. Diese Analyse soll dazu beitragen, ein besseres Verständnis für das Verhalten leichtgläubiger Twitter-Nutzer zu gewinnen und mögliche Implikationen für die Informationsverbreitung und das soziale Online-Verhalten aufzuzeigen \cite{twitter2012twitter}.
	
	\section{Hintergrund}
	Hier stellen Sie den theoretischen Hintergrund für Ihre Arbeit dar und führen relevante Literaturquellen an.
	
	\section{Methodik}
	Beschreiben Sie Ihre Forschungsmethodik im Detail, einschließlich Datenerhebung und Analyseverfahren.
	
	\section{Ergebnisse}
	Präsentieren Sie die Ergebnisse Ihrer Forschung. Verwenden Sie Grafiken, Diagramme und Tabellen zur Veranschaulichung.
	
	\section{Diskussion}
	Diskutieren Sie Ihre Ergebnisse im Kontext der Literatur und bewerten Sie die Bedeutung Ihrer Ergebnisse.
	
	\section{Schlussfolgerung}
	Fassen Sie die wichtigsten Erkenntnisse zusammen und geben Sie einen Ausblick auf mögliche zukünftige Forschungsrichtungen.
	
	\section{Literaturverzeichnis}
	Fügen Sie alle verwendeten Quellen im Literaturverzeichnis hinzu, einschließlich des oben genannten Zitats. Hier ist ein einfaches Beispiel:
	
	\cite{twitter2012twitter}
	
	\bibliographystyle{plain} 
	\bibliography{literatur}  
	\appendix
	\section{Anhang}
	Hier können Sie zusätzliche Informationen wie Rohdaten, Code oder weitere Grafiken anfügen.
	zitat
\end{document}
